\documentclass[]{book}
\usepackage{lmodern}
\usepackage{amssymb,amsmath}
\usepackage{ifxetex,ifluatex}
\usepackage{fixltx2e} % provides \textsubscript
\ifnum 0\ifxetex 1\fi\ifluatex 1\fi=0 % if pdftex
  \usepackage[T1]{fontenc}
  \usepackage[utf8]{inputenc}
\else % if luatex or xelatex
  \ifxetex
    \usepackage{mathspec}
  \else
    \usepackage{fontspec}
  \fi
  \defaultfontfeatures{Ligatures=TeX,Scale=MatchLowercase}
\fi
% use upquote if available, for straight quotes in verbatim environments
\IfFileExists{upquote.sty}{\usepackage{upquote}}{}
% use microtype if available
\IfFileExists{microtype.sty}{%
\usepackage{microtype}
\UseMicrotypeSet[protrusion]{basicmath} % disable protrusion for tt fonts
}{}
\usepackage[margin=1in]{geometry}
\usepackage{hyperref}
\hypersetup{unicode=true,
            pdftitle={Tests multivariados de independencia basados en rangos de los vecinos más cercanos},
            pdfauthor={Hernán David Torres, Olga Lisethe Castellanos},
            pdfborder={0 0 0},
            breaklinks=true}
\urlstyle{same}  % don't use monospace font for urls
\usepackage{natbib}
\bibliographystyle{apalike}
\usepackage{longtable,booktabs}
\usepackage{graphicx,grffile}
\makeatletter
\def\maxwidth{\ifdim\Gin@nat@width>\linewidth\linewidth\else\Gin@nat@width\fi}
\def\maxheight{\ifdim\Gin@nat@height>\textheight\textheight\else\Gin@nat@height\fi}
\makeatother
% Scale images if necessary, so that they will not overflow the page
% margins by default, and it is still possible to overwrite the defaults
% using explicit options in \includegraphics[width, height, ...]{}
\setkeys{Gin}{width=\maxwidth,height=\maxheight,keepaspectratio}
\IfFileExists{parskip.sty}{%
\usepackage{parskip}
}{% else
\setlength{\parindent}{0pt}
\setlength{\parskip}{6pt plus 2pt minus 1pt}
}
\setlength{\emergencystretch}{3em}  % prevent overfull lines
\providecommand{\tightlist}{%
  \setlength{\itemsep}{0pt}\setlength{\parskip}{0pt}}
\setcounter{secnumdepth}{5}
% Redefines (sub)paragraphs to behave more like sections
\ifx\paragraph\undefined\else
\let\oldparagraph\paragraph
\renewcommand{\paragraph}[1]{\oldparagraph{#1}\mbox{}}
\fi
\ifx\subparagraph\undefined\else
\let\oldsubparagraph\subparagraph
\renewcommand{\subparagraph}[1]{\oldsubparagraph{#1}\mbox{}}
\fi

%%% Use protect on footnotes to avoid problems with footnotes in titles
\let\rmarkdownfootnote\footnote%
\def\footnote{\protect\rmarkdownfootnote}

%%% Change title format to be more compact
\usepackage{titling}

% Create subtitle command for use in maketitle
\newcommand{\subtitle}[1]{
  \posttitle{
    \begin{center}\large#1\end{center}
    }
}

\setlength{\droptitle}{-2em}
  \title{Tests multivariados de independencia basados en rangos de los vecinos
más cercanos}
  \pretitle{\vspace{\droptitle}\centering\huge}
  \posttitle{\par}
  \author{Hernán David Torres, Olga Lisethe Castellanos}
  \preauthor{\centering\large\emph}
  \postauthor{\par}
  \predate{\centering\large\emph}
  \postdate{\par}
  \date{2017-11-17}

\usepackage{booktabs}
\usepackage{amsthm}
\makeatletter
\def\thm@space@setup{%
  \thm@preskip=8pt plus 2pt minus 4pt
  \thm@postskip=\thm@preskip
}
\makeatother

\begin{document}
\maketitle

{
\setcounter{tocdepth}{1}
\tableofcontents
}
\chapter{INTRODUCCIÓN}\label{introduccion}

Se exhiben, a continuación, algunos tests multivariados de independencia
entre dos vectores aleatorios de dimensión arbitraria usados,
convenientemente, en situaciones de \emph{\emph{HDLSSD}}.
\citep{sarkar2017some}

\chapter{HDLSSD}\label{intro}

Previo a lo anterior, es importante mencionar que \emph{\emph{High
Dimensional Data Low Sample Size}} \emph{\emph{HDLSSD}} refiere, como su
nombre lo indica a las situaciones en las cuales se tiene alta
dimensionalidad (gran cantidad de variables) y tamaños muestrales
pequeños.

\chapter{PROBLEMA}\label{problema}

Se han desarrollado muchos tests estadísticos tanto paramétricos como no
paramétricos para probar independencia entre 2 vectores aleatorios, pero
dichos test no son aplicables en casos de alta dimensionalidad y
muestras pequeñas.

\section{DESARROLLO TEÓRICO}\label{desarrollo-teorico}

En las situaciones tratadas, se consideran \emph{\emph{n}} relizaciones
independientes

\[
 \mathbf{\begin{equation} z_{1} =
\begin{pmatrix} x_1\\
y_1\\
\end{pmatrix} 
 ,  z_{2} =
\begin{pmatrix} x_2\\
y_2\\
\end{pmatrix}, ... ,    z_{n} =
\begin{pmatrix} x_n\\
y_n\\
\end{pmatrix}
\end{equation} }
\]

de un vector aleatorio continuo
\(\begin{equation} Z = \mathbf{\begin{pmatrix} X\\ Y\\ \end{pmatrix}} \ donde \ \mathbf{X} \in \chi \subseteq \mathbb{R}^p \ y \ \mathbf{Y} \in \chi \subseteq \mathbb{R}^q \end{equation}\)
,

\section{TEST EXISTENTES}\label{test-existentes}

Entre los dichos test podemos enumerar, para el caso paramétrico:

\begin{itemize}
\tightlist
\item
  Test de máxima verosimilitud basado en el estadístico Wilk's
\item
  Test de prueba de la raíz más grande de Roy
\item
  Test de la traza de Hotelling-Lawley
\item
  Test de la traza de Pillai-Bartlett
\end{itemize}

Para el caso no paramétrico:

\emph{Univariados:}

\begin{itemize}
\tightlist
\item
  Estadístico de Spearman \(\rho\) y Estadístico de Kendall \(\tau\)
\item
  Test basado en cuadrantes estadísticos
\end{itemize}

\emph{Multivariados:}

\begin{itemize}
\tightlist
\item
  Test basados en\\
\item
  Extensión del test de cuadrantes para mayores dimensiones mediantes
  interdirecciones
\item
  Generalizaciones multivariadas de los test basados en Spearman's,
  Kendall's \(\tau\) y cuadrante estadístico usando signos espaciales y
  rangos.
\end{itemize}

Dada la necesidad de manipular datos de este tipo en distintas áreas de
investigación, se han desarrollado algunos test para probar
independencia entre dos vectores, basados en distancias euclidianas
entre punto, denotando:

\[ \begin{equation} 
d_{ij}^x = d_x(x_i, x_j)   
\end{equation} \]

,y

\[ \begin{equation} 
d_{ij}^y = d_y(y_i, y_j)
\end{equation} \]

\bibliography{book.bib,packages.bib,referencias.bib}


\end{document}
